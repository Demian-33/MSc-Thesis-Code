\thispagestyle{plain}

\phantomsection
\addcontentsline{toc}{chapter}{RESUMEN}

\begin{center}
{\bfseries
ESTIMACIÓN DE LOS MODELOS DE REGRESIÓN LOG-NORMAL SESGADO Y PROBIT SESGADO LATENTE EN ÁREAS PEQUEÑAS MEDIANTE INFERENCIA BAYESIANA VARIACIONAL}

\begin{spacing}{1.0}
Saul Arturo Ortiz Muñoz, M.C. \\
Colegio de Postgraduados, 2026
\end{spacing}

RESUMEN
\end{center}

Se proponen tres modelos de regresión Bayesiana variacional en áreas pequeñas con errores normales asimétricos: uno para respuesta continua y otro para respuesta binaria/ordinal. En estos dos últimos, el ajuste es posible gracias al uso de variables latentes. Se siguió un enfoque Bayesiano objetivo: se empleó la distribución de referencia para los parámetros de forma de la distribución normal sesgada y una \textit{a priori} para realizar búsqueda estocástica de variables. Al resto de parámetros se asignaron densidades planas. Se desarrollaron estudios de simulación para cada modelo y se comparó contra el método Hamiltoniano Monte Carlo, basado en \textit{Markov chain Monte Carlo}. Se encontró que el método variacional es bastante competitivo cuando se ajusta a respuestas continuas. Se estudió un conjunto de datos sobre el ingreso corriente total per cápita en los hogares de la Ciudad de México, con información proveniente de fuentes oficiales y siguiendo los criterios de procesamiento establecidos por el Coneval. El modelo continuo señala que la variable respuesta presenta distintos grados de sesgo por alcaldía. Se identificaron las covariables más relevantes para los modelos de regresión. Se concluye que el método variacional es una alternativa viable para acelerar la inferencia sin pérdidas abrumadores de precisión, con respecto a los algoritmos Bayesianos usuales basados en muestreo. La aproximación variacional mostró ser especialmente útil con las respuestas continuas.

%\hspace{1cm}
%\vfill
\noindent\textbf{Palabras clave:} Distribución normal asimétrica, estimación en áreas pequeñas,  inferencia Bayesiana variacional, regresión probit.

\newpage

\thispagestyle{plain}
\phantomsection
\addcontentsline{toc}{chapter}{ABSTRACT}

\begin{center}
{\bfseries
ESTIMATION OF SKEWED LOG-NORMAL AND LATENT SKEWED PROBIT REGRESSION MODELS IN SMALL AREAS USING VARIATIONAL BAYESIAN INFERENCE}

\begin{spacing}{1.0}
Saul Arturo Ortiz Muñoz, M.C. \\
Colegio de Postgraduados, 2026
\end{spacing}

ABSTRACT
\end{center}


%\hspace{1cm}
%\vfill
\noindent\textbf{Key words:} skew-normal distribution, small-area estimation, variational bayes, probit regression. 

\newpage

\thispagestyle{plain}

\chapter*{AGRADECIMIENTOS}
%\addcontentsline{toc}{chapter}{DEDICATORIA}

A la Secretaría de Ciencia, Humanidades, Tecnología e Innovación (SECIHTI) por la ayuda financiera.

Al Colegio de Postgraduados, por brindarme las facilidades y por la educación recibida. Particularmente, al campus Montecillo, por contribuir de manera importante en mi formación y desarrollo profesional.

A los miembros del consejo particular y sinodal interno, gracias por sus comentarios valiosos y acertados.

A mi familia, por permitirme seguir estudiando y por su apoyo en todos estos años.


\newpage
\thispagestyle{plain}


%\chapter*{DEDICATORIA}
%\addcontentsline{toc}{chapter}{DEDICATORIA}

%\hfill\vfill\hfill c:

\hfill\textit{El regalo más hermoso es el perdón, lo más maravilloso del mundo es el amor, la felicidad mas dulce es la paz.}



