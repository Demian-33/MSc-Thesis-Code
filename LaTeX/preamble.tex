\usepackage{amsmath, amssymb, mathtools, tcolorbox, algorithm2e, xcolor, geometry, hyperref, float, caption, subcaption, cancel, tabularx, multirow, makecell, rotating, xltabular, siunitx, lipsum, paracol, titlesec, fontspec, tocloft, minted, fancyhdr, emptypage, pdfpages, setspace, titlesec, enumitem}

%% Quebrar formulas
\allowdisplaybreaks

%% Interlineado
%\renewcommand{\baselinestretch}{1.5} % de forma global
\setstretch{1.5} % solo al texto
\setlength{\parskip}{10pt} % distancia de 10pt entre párrafos
\setlength{\parindent}{0pt}
%% En listas (o usando enumitem)
%\setlength{\parsep}{0pt}
%\setlength{\topsep}{0pt}
%\setlength{\partopsep}{0pt}
\setlist{itemsep=-6pt, topsep=0pt} % \setlist[itemize]{...}

\captionsetup{labelfont={bf,small}, textfont=small, singlelinecheck=false, justification=centering}

\titleformat{\chapter}[block]% 
{\normalfont\bfseries\centering}{CAPÍTULO \thechapter. }{0pt}{}
\titleformat{\section}%
{\normalfont\bfseries}{\thesection}{10pt}{}{}
\titleformat{\subsection}%
{\normalfont\bfseries}{\thesubsection}{10pt}{}{}

\titlespacing{\chapter}{0pt}{10pt}{10pt}
\titlespacing{\section}{0pt}{10pt}{10pt}
\titlespacing{\subsection}{0pt}{10pt}{10pt}
\titlespacing{\subsubsection}{0pt}{10pt}{10pt}

%% Redefinir figura para añadir 10pt antes y después
%\BeforeBeginEnvironment{figure}{\vspace{10pt}}
%\AfterEndEnvironment{figure}{\vspace{10pt}}

% Redefinir tabla para añadir 10pt antes y después
%\BeforeBeginEnvironment{table}{\vspace{10pt}}
%\AfterEndEnvironment{table}{\vspace{10pt}}

% Redefinir algoritmo para añadir 10pt antes y después
\BeforeBeginEnvironment{algorithm}{\vspace{10pt}}
\AfterEndEnvironment{algorithm}{\vspace{10pt}}

\usepackage[font=small]{caption}

%% Nombre de capitulos, secciones, subsecciones
\renewcommand{\chaptername}{CAPÍTULO}
\renewcommand{\chapterautorefname}{Capítulo}
\renewcommand{\subsectionautorefname}{Sección}
\renewcommand{\subsectionautorefname}{Sección}
\renewcommand{\subsubsectionautorefname}{Sección}
\renewcommand{\sectionautorefname}{Sección}
\renewcommand{\subsectionautorefname}{Sección}
\renewcommand{\equationautorefname}{Ecuación}
\renewcommand{\figureautorefname}{Figura}
\renewcommand{\tableautorefname}{Cuadro}


\renewcommand{\figurename}{Figura}
\renewcommand{\tablename}{Cuadro}

\renewcommand{\listfigurename}{\normalsize\bfseries \hfill LISTA DE FIGURAS \hfill}
\renewcommand{\listtablename}{\normalsize\bfseries \hfill LISTA DE CUADROS \hfill}

\renewcommand{\contentsname}{CONTENIDO}
\renewcommand{\cfttoctitlefont}{\normalsize\bfseries\hfill}
\renewcommand{\cftaftertoctitle}{\hfill\hfill} 

%% negritas en lista de figuras y cuadros
\renewcommand{\cftfignumwidth}{7em}
\renewcommand{\cftfigpresnum}{\bfseries Figura }
\renewcommand{\cfttabnumwidth}{7em}
\renewcommand{\cfttabpresnum}{\bfseries Cuadro }

%% Espacio despues de Contenidos
\setlength{\cftbeforetoctitleskip}{-14pt}  % Antes de "CONTENIDO"
\setlength{\cftaftertoctitleskip}{10pt}    % Después de "CONTENIDO"
\setlength{\cftbeforechapskip}{0pt} % espacio entre capitulos?

\setlength{\cftbeforeloftitleskip}{-14pt}  % Antes de "LISTA DE FIGURAS"
\setlength{\cftafterloftitleskip}{10pt}    % Después de "LISTA DE FIGURAS"

\setlength{\cftbeforelottitleskip}{-14pt}  % Antes de "LISTA DE CUADROS"
\setlength{\cftafterlottitleskip}{10pt}    % Después de "LISTA DE CUADROS"

% Agrega puntos de relleno (líneas punteadas) a las entradas manuales del TOC
\renewcommand{\cftdotsep}{1}                % Puntos más juntos
\renewcommand{\cftchapleader}{\cftdotfill{\cftdotsep}} % Líneas punteadas en capítulos y similares
% Elimina la sangría de los capítulos en el TOC
\renewcommand{\cftchapindent}{0pt}
\renewcommand{\cftchapnumwidth}{0pt}


%% minted
% \usepackage[outputdir=./]{minted}
% \usepackage[cache=false]{minted}

%https://github.com/gpoore/minted/issues/162
% default mintinline no funciona dentro de tabularx o xltabular, ahi se propone usar myinline definido asi:
% nota: usar XeLaTeX, por el comando de abajo supongo...
\RequireXeTeX
\makeatletter
\def\myinline#1#2{%
\ifx\@footnotetext\TX@trial@ftn
\detokenize{#2}%
\else
\mintinline{#1}{#2}%
\fi}
\makeatother

\newcommand{\stan}[1]{\mintinline{stan}{#1}}

\tcbuselibrary{most}
\tcbset{
 breakable = true,
}

%\usepackage{marginnote}
% \geometry{
%     paper = letterpaper,
%     % textwidth=22.5cm, %toma todo el espacio que resta
%     right=3.9cm,
%     left=3.9cm,
%     % margin = 2cm,
%     marginparwidth=90pt,
%     top=20mm,
%     bottom=15mm,
%     % showframe
% }

\geometry{
    paper = letterpaper,
    margin = 2.54cm,
    % textwidth=22.5cm, %toma todo el espacio que resta
    %showframe
}


\DeclareMathOperator{\supp}{supp}
\DeclareMathOperator*{\argmin}{arg\,min}
\DeclareMathOperator*{\argmax}{arg\,max}
\DeclareMathOperator*{\diag}{diag}
\DeclareMathOperator*{\tr}{tr}

% The * in \DeclareMathOperator* places the underscored argument underneath the word rather than to the bottom right of it.

% \def\bcancelto#1#2{\let\canto@vector\cantox@vector\cancelto{#1}{#2}}

\newcommand{\KL}[2]{\operatorname{KL}\left(#1\Vert#2\right)}
\newcommand{\ELBO}[1]{\operatorname{ELBO}\left(#1\right)}
\newcommand{\E}{\operatorname{\mathbb{E}}}
\newcommand{\Var}{\operatorname{\mathbb{V}ar}}
\newcommand{\Cov}{\operatorname{\mathbb{C}ov}}

\definecolor{mygray}{gray}{0.95}

\newcommand{\code}[2][mygray]{\texttt{\colorbox{#1}{#2}}}

\newcommand{\enfasis}[2][mygray]{\colorbox{#1}{#2}}

\newcommand{\Dir}[2]{\operatorname{Dir}\left(#1\vert#2\right)}

\newcommand{\siunitxcolnames}[0]{{\bfseries Par.} & {\bfseries Valor real} & {\bfseries Media p.} & {\bfseries I.C. 2.5\%} & {\bfseries I.C. 97.5\%} & {\bfseries Media p.} & {\bfseries I.C. 2.5\%} & {\bfseries I.C. 97.5\%}}

\newcommand{\siunitxcolnamess}[0]{{\bfseries Par.} & {\bfseries Media p.} & {\bfseries I.C. 2.5\%} & {\bfseries I.C. 97.5\%} & {\bfseries Media p.} & {\bfseries I.C. 2.5\%} & {\bfseries I.C. 97.5\%}}

\newcommand{\siunitxcolnamesss}[0]{{\bfseries Par.} & {\bfseries Media p.} & {\bfseries I.C. 2.5\%} & {\bfseries I.C. 97.5\%}}

\newcommand{\siunitxcolnameszero}[0]{ & {\bfseries Media p.} & {\bfseries I.C. 2.5\%} & {\bfseries I.C. 97.5\%} & {\bfseries Media p.} & {\bfseries I.C. 2.5\%} & {\bfseries I.C. 97.5\%} & {$\hat{R}$}}

\newcommand{\siunitxcolnamesi}[0]{{\bfseries Par.} & {\bfseries Media p.} & {\bfseries I.C. 2.5\%} & {\bfseries I.C. 97.5\%} & {\bfseries Media p.} & {\bfseries I.C. 2.5\%} & {\bfseries I.C. 97.5\%} & {$\hat{R}$}}

\newcommand{\siunitxcolnamesii}[0]{{$\boldsymbol{\beta}$} & {\bfseries Media p.} & {\bfseries I.C. 2.5\%} & {\bfseries I.C. 97.5\%} & {\bfseries Media p.} & {\bfseries I.C. 2.5\%} & {\bfseries I.C. 97.5\%} & {$\hat{R}$}}

\newcolumntype{O}{>{}S[table-format=2.1, round-mode=places, detect-weight, round-precision=1]}

\newcolumntype{Q}{>{}S[table-format=1.3, round-mode=places, detect-weight, round-precision=3]}

\newcolumntype{T}{>{}S[table-format=6.3]}

\newcolumntype{R}{>{}S[table-format=3.3, detect-weight]}

\newcolumntype{V}{>{}S[table-format=5.3, detect-weight]}

\newcolumntype{I}{>{}S[table-format=3.0, detect-weight]}


% https://tex.stackexchange.com/questions/318372/align-numbers-by-decimal-point-using-siunitx-and-bfseries


\newcommand{\colnames}[0]{Par. & Valor real & Media p. & I.C. 2.5\% & I.C. 97.5\% & Media p. & I.C. 2.5\% & I.C. 97.5\%}


%% Algoritmos
\renewcommand{\algorithmcfname}{Algoritmo}
\renewcommand{\algorithmautorefname}{Algoritmo}
\newcommand\mycommfont[1]{\normalsize\ttfamily\textcolor{blue}{#1}}
\SetCommentSty{mycommfont}
\RestyleAlgo{ruled}
\SetKwInput{kwInit}{Initialize}
%\renewcommand\listoflistingscaption{Lista de códigos fuente}
%\listoflistings

%% Fuentes
% \usepackage{fontspec}
% \setmainfont{Times New Roman}

%% Modificar capítulos
%\titleformat{\chapter}{\centering\Large\bfseries\scshape}{\thechapter}{1em}{}

%https://www.reddit.com/r/LaTeX/comments/vnn7uu/help_sans_bf_sc_fonts/

%% hypersetup
\hypersetup{
  pdftitle=Estimación de los modelos de regresión log-normal sesgado y probit sesgado latente en áreas pequeñas mediante inferencia Bayesiana variacional,
  pdfauthor=Saul Arturo Ortiz Muñoz,
  pdfsubject=Tesis de maestría,
  colorlinks=true,
  citecolor=blue,
  filecolor=blue,
  linkcolor=blue,
  urlcolor=blue}

%% Modificar TOC
%% 1. INTRODUCCIÓN a CAPITULO 1. INTRODUCCIÓN

\makeatletter
\def\@chapter[#1]#2{%
  \phantomsection %
  \ifnum \c@secnumdepth >\m@ne
    \refstepcounter{chapter}%
    \typeout{\@chapapp\space\thechapter.}%
   \addcontentsline{toc}{chapter}{\texorpdfstring{\MakeUppercase{CAPÍTULO \thechapter.} \ #1}{CAPÍTULO \thechapter. \ #1}}%
  \else
    \addcontentsline{toc}{chapter}{#1}%
  \fi
  \chaptermark{#1}%
  \if@twocolumn
    \@topnewpage[\@makechapterhead{#2}]%
  \else
    \cleardoublepage
    \@makechapterhead{#2}%
    \@afterheading
  \fi}
\makeatother

%% Encabezados

%\fancypagestyle{plain}{
%  \fancyhf{} % clear all header and footer fields
%  \fancyfoot[C]{\thepage} % except the center page number
%  \renewcommand{\headrulewidth}{0pt}
%  \renewcommand{\footrulewidth}{0pt}
%}
\pagestyle{fancy}
\fancyhf{}
%\nouppercase{foo}
%\MakeUppercase{foo}
%% use with twoside
%\fancyhead[LE]{{\leftmark}} % left header on even pages
%\fancyhead[RO]{\nouppercase{\rightmark}} % right header on odd pages
%% use with oneside
%% esta chisyoso, en oneside, chapter puede iniciar en par o impar, entonces los encabezados no son consistentes
%\fancyhead[R]{\MakeUppercase{\ifodd\value{page}\else\leftmark\fi}}
%\fancyhead[L]{\MakeUppercase{\ifodd\value{page}\rightmark\else\fi}}
% \fancyhead[L]{O}
\fancyhead[C]{}
% \fancyhead[R]{}
% \renewcommand{\headrulewidth}{0.4pt}
% \renewcommand{\footrulewidth}{0.4pt}
\fancyfoot[L]{}
\fancyfoot[C]{\thepage}
\fancyfoot[R]{}

%% Evitar hypenation?
\tolerance=1
\emergencystretch=\maxdimen
\hyphenpenalty=10000
\exhyphenpenalty=10000
\hbadness=10000